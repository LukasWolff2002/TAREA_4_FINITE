\documentclass{article}  % Define la clase del documento.

% Paquetes de idioma y codificación
\usepackage[utf8]{inputenc}
\usepackage[T1]{fontenc}
\usepackage[spanish]{babel}  % Ajusta el idioma del documento a español.
\addto\captionsspanish{
  \renewcommand{\figurename}{Figure}
  \renewcommand{\tablename}{Table}
}

\usepackage{tabularx}  % Permite la creación de tablas con ancho ajustable.

\usepackage{caption}
\usepackage{subcaption}

% Paquete de geometría para configurar márgenes y tamaño de papel
\usepackage[letterpaper, margin=3cm]{geometry}

% Paquetes de tipografía
\usepackage{mathptmx}    % Usa Times New Roman como fuente.
\usepackage{microtype}   % Mejora la justificación del texto.

% Paquetes para manejo de colores y gráficos
\usepackage{xcolor}      % Define y utiliza colores.
\usepackage{graphicx}    % Permite la inserción de imágenes.
\usepackage{tikz}        % Creación de gráficos vectoriales.

% Configuración de enlaces y referencias cruzadas
\usepackage{hyperref}
\hypersetup{
    colorlinks   = true,
    linkcolor    = darkblue,
    citecolor    = black,
    filecolor    = blue,
    urlcolor     = blue
}

\usepackage{media9} % Permite la inserción de multimedia.

% Paquetes para la mejora visual de tablas y figuras
\usepackage{booktabs}    % Para tablas de alta calidad.
\usepackage{float}       % Controla la posición de figuras y tablas.

% Paquete para la personalización de códigos fuente
\usepackage{listings}
\lstset{
    literate=
    {á}{{\'a}}1 {é}{{\'e}}1 {í}{{\'i}}1 {ó}{{\'o}}1 {ú}{{\'u}}1
    {Á}{{\'A}}1 {É}{{\'E}}1 {Í}{{\'I}}1 {Ó}{{\'O}}1 {Ú}{{\'U}}1
    {ñ}{{\~n}}1 {Ñ}{{\~N}}1 {ü}{{\"u}}1 {Ü}{{\"U}}1,
    backgroundcolor=\color{backcolour},
    commentstyle=\color{codegreen},
    keywordstyle=\color{codepurple},
    numberstyle=\tiny\color{codegray},
    stringstyle=\color{red},
    basicstyle=\ttfamily\small,
    breakatwhitespace=false,
    breaklines=true,
    captionpos=b,
    keepspaces=true,
    numbers=left,
    numbersep=5pt,
    showspaces=false,
    showstringspaces=false,
    showtabs=false,
    tabsize=2,
    language=TeX,
    morecomment=[l]\#,
    frame=single,
    rulecolor=\color{black}
}

% Definición de colores al estilo Visual Studio Code
\definecolor{darkblue}{rgb}{0.0, 0.0, 0.55}  % Enlaces
\definecolor{codegreen}{rgb}{0.25, 0.49, 0.48}  % Comentarios
\definecolor{codegray}{rgb}{0.5, 0.5, 0.5}  % Números y anotaciones
\definecolor{codepurple}{rgb}{0.58, 0, 0.82}  % Palabras clave
\definecolor{backcolour}{rgb}{0.95, 0.95, 0.92}  % Fondo de código

% Configuraciones de párrafo y matemáticas
\usepackage{amsmath}
\usepackage{parskip}    % Espaciado entre párrafos.
\usepackage{ragged2e}   % Justificación mejorada.
\usepackage{multicol}

% Configuración de secciones y encabezados
\usepackage{titlesec}
\titleclass{\part}{top} % Make part like a class
\titleformat{\part}[display]
  {\normalfont\huge\bfseries\centering}{\thepart}{40pt}{\Huge}
\titlespacing*{\part}{0pt}{-60pt}{10pt}
\titleformat{\part}
  {\normalfont\huge\bfseries}{}{0pt}{}

% Asegúrate de usar esto para mantener el estilo en las páginas de las partes
\titleformat{\part}[display]
  {\normalfont\huge\bfseries}{}{0pt}{}
  [\thispagestyle{fancy}] % Aplica el estilo fancy a las páginas de las partes

% Configuración de encabezados y pies de página personalizados
\usepackage{fancyhdr}
\pagestyle{fancy}
\fancyhf{}
\fancyhead[L]{\raisebox{0.20cm}{\textbf{Finite Elements - IOC5107}}}
\fancyhead[R]{\raisebox{0.1cm}{\includegraphics[width=0.25\linewidth]{LOGO_UNIVERSIDAD.jpg}}}
\fancyhead[C]{\rule{\textwidth}{0.6pt}}
\fancyfoot[C]{\rule{\textwidth}{0.6pt}}
\fancyfoot[R]{\raisebox{-1.5\baselineskip}{\thepage}}
\renewcommand{\headrulewidth}{0pt}
\renewcommand{\footrulewidth}{0pt}

% Configuración avanzada de geometría
\geometry{
  top=3.5cm, % Aumenta el espacio en la parte superior para subir el encabezado
  bottom=2.5cm,
  headheight=2.5cm % Aumenta la altura del encabezado si es necesario
}

% Configuracion de bibliografia
\usepackage{natbib}
\bibliographystyle{unsrtnat}  % Puedes cambiarlo por `unsrtnat`, `abbrvnat`, etc.

\begin{document}
%----------------------------------------------------------------------------------------
% PORTADA
%----------------------------------------------------------------------------------------
\begin{titlepage}%Inicio de la carátula, solo modificar los datos necesarios
\newcommand{\HRule}{\rule{\linewidth}{0.5mm}} 
\center 
%----------------------------------------------------------------------------------------
%	ENCABEZADO
%----------------------------------------------------------------------------------------
\includegraphics[width=10cm]{LOGO_UNIVERSIDAD.jpg}\\ % Si esta plantilla se copio correctamente, va a llevar la imagen del logo de la facultad.OBS: Es necesario incluir el paquete: graphicx
\vspace{3cm}
%----------------------------------------------------------------------------------------
%	SECCION DEL TITULO
%----------------------------------------------------------------------------------------
\HRule \\[0.4cm]
{ \huge \bfseries Finite Elements - IOC5107}\\[0.4cm] % Titulo del documento
{ \huge \bfseries Final Report Homework 2}\\[0.4cm] % Titulo del documento
\HRule \\[1.5cm]
 \vspace{5cm}
%----------------------------------------------------------------------------------------
%	SECCION DEL AUTOR
%----------------------------------------------------------------------------------------
\begin{flushright}
  { \textbf{Profesor:}\\
  José Antonio Abell\\
  \textbf{Ayudante:}\\
  Nicolás Mora\\
  \textbf{Alumnos:} \\
  Felipe Vicencio\\
  Lukas Wolff\\
}
\end{flushright}
\vspace{1cm}
%----------------------------------------------------------------------------------------
%	SECCION DE LA FECHA
%----------------------------------------------------------------------------------------
{\large \textbf{\today}}\\[2cm] % El comando \today coloca la fecha del dia, y esto se actualiza con cada compilacion, en caso de querer tener una fecha estatica, reemplazar el \today por la fecha deseada
\end{titlepage}

\newpage
\thispagestyle{empty} % Deshabilita el número de página en la página del índice

%----------------------------------------------------------------------------------------
%  INDICE
%----------------------------------------------------------------------------------------
%\newpage
%\thispagestyle{empty} % Deshabilita el número de página en la página del índice
%\tableofcontents
%\thispagestyle{plain} % Deshabilita el encabezado en la página del índice
%\thispagestyle{empty} % Deshabilita el número de página en la página del índice
%\newpage

%\newpage
%\thispagestyle{empty}
%\listoffigures 
%\thispagestyle{plain} % Deshabilita el encabezado en la página del índice %
%\thispagestyle{empty}
\newpage
%----------------------------------------------------------------------------------------
%ACÁ EMPIEZA EL INFORME
\setcounter{page}{1}
%----------------------------------------------------------------------------------------

Podemos decir esto

Si:

\begin{equation}
  u(x) = ||x||^\alpha
\end{equation}

Podemos decir que:

\begin{equation}
  |u - u_h|^2 = |(||x||^\alpha)|^2 - u K u \leq C h \quad \text{h no es cte}
\end{equation}

Donde x es el vector fuerzas y u es el vector desplazamientos.

Por lo tanto, podemos decir que:

\begin{equation}
  x - uKu \leq C h \quad \text{cuanto vale h?}
\end{equation}

\section{Calculo de largo del elemento}

Para calcular el largo del elemento, podemos usar la siguiente fórmula:

\begin{equation}
  S1 = \frac{L(1-r)}{1-r^n}
\end{equation}

Donde L es 1, ya que es el largo

r = 1.3 o 1 dividido en 1.3 dependiendo de la direccion. De arriba a abajo es 1 dividido en 1.3 y viceversa

N = num of divicion - 1

Luego se cumple que:

\begin{equation}
  S_{i+1} = S_i \cdot r
\end{equation}

Hagamos la integracion

\begin{equation}
\int_{x_j}^{x_{j+1}} (x - x_j)\, dx 
= \frac{x_{j+1}^2 - 2x_{j+1} x_j + x_j^2}{2}
= \frac{1}{2} \left(
\left( 
\frac{L (1 - r)}{1 - r^n} \cdot \frac{1}{1 + |x_j| \cdot 10^{16}} 
+ 
\frac{(x_{j+1} - x_j) \cdot x_j}{x_j + 10^{-16}} \cdot r
\right)^2
\right)
\end{equation}



\end{document}