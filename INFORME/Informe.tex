\documentclass{article}  % Define la clase del documento.

% Paquetes de idioma y codificación
\usepackage[utf8]{inputenc}
\usepackage[T1]{fontenc}
\usepackage[spanish]{babel}  % Ajusta el idioma del documento a español.
\addto\captionsspanish{
  \renewcommand{\figurename}{Figure}
  \renewcommand{\tablename}{Table}
}

\usepackage{tabularx}  % Permite la creación de tablas con ancho ajustable.

\usepackage{caption}
\usepackage{subcaption}

% Paquete de geometría para configurar márgenes y tamaño de papel
\usepackage[letterpaper, margin=3cm]{geometry}

% Paquetes de tipografía
\usepackage{mathptmx}    % Usa Times New Roman como fuente.
\usepackage{microtype}   % Mejora la justificación del texto.

% Paquetes para manejo de colores y gráficos
\usepackage{xcolor}      % Define y utiliza colores.
\usepackage{graphicx}    % Permite la inserción de imágenes.
\usepackage{tikz}        % Creación de gráficos vectoriales.

% Configuración de enlaces y referencias cruzadas
\usepackage{hyperref}
\hypersetup{
    colorlinks   = true,
    linkcolor    = darkblue,
    citecolor    = black,
    filecolor    = blue,
    urlcolor     = blue
}

\usepackage{media9} % Permite la inserción de multimedia.

% Paquetes para la mejora visual de tablas y figuras
\usepackage{booktabs}    % Para tablas de alta calidad.
\usepackage{float}       % Controla la posición de figuras y tablas.

% Paquete para la personalización de códigos fuente
\usepackage{listings}
\lstset{
    literate= {á}{{\'a}}1 {é}{{\'e}}1 {í}{{\'i}}1 {ó}{{\'o}}1 {ú}{{\'u}}1
    {Á}{{\'A}}1 {É}{{\'E}}1 {Í}{{\'I}}1 {Ó}{{\'O}}1 {Ú}{{\'U}}1
    {ñ}{{\~n}}1 {Ñ}{{\~N}}1 {ü}{{\"u}}1 {Ü}{{\"U}}1,
    backgroundcolor=\color{backcolour},
    commentstyle=\color{codegreen},
    keywordstyle=\color{codepurple},
    numberstyle=\tiny\color{codegray},
    stringstyle=\color{red},
    basicstyle=\ttfamily\small,
    breakatwhitespace=false,
    breaklines=true,
    captionpos=b,
    keepspaces=true,
    numbers=left,
    numbersep=5pt,
    showspaces=false,
    showstringspaces=false,
    showtabs=false,
    tabsize=2,
    language=TeX,
    morecomment=[l]\#,
    frame=single,
    rulecolor=\color{black}
}

% Definición de colores al estilo Visual Studio Code
\definecolor{darkblue}{rgb}{0.0, 0.0, 0.55}  % Enlaces
\definecolor{codegreen}{rgb}{0.25, 0.49, 0.48}  % Comentarios
\definecolor{codegray}{rgb}{0.5, 0.5, 0.5}  % Números y anotaciones
\definecolor{codepurple}{rgb}{0.58, 0, 0.82}  % Palabras clave
\definecolor{backcolour}{rgb}{0.95, 0.95, 0.92}  % Fondo de código

% Configuraciones de párrafo y matemáticas
\usepackage{amsmath}   % Para mejoras en el entorno matemático
\usepackage{amssymb}   % Incluye \mathbb y otros símbolos
\usepackage{parskip}    % Espaciado entre párrafos.
\usepackage{ragged2e}   % Justificación mejorada.
\usepackage{multicol}

% Configuración de secciones y encabezados
\usepackage{titlesec}
\titleclass{\part}{top} % Make part like a class
\titleformat{\part}[display]
  {\normalfont\huge\bfseries\centering}{\thepart}{40pt}{\Huge}
\titlespacing*{\part}{0pt}{-60pt}{10pt}
\titleformat{\part}
  {\normalfont\huge\bfseries}{}{0pt}{}

% Asegúrate de usar esto para mantener el estilo en las páginas de las partes
\titleformat{\part}[display]
  {\normalfont\huge\bfseries}{}{0pt}{}
  [\thispagestyle{fancy}] % Aplica el estilo fancy a las páginas de las partes

% Configuración de encabezados y pies de página personalizados
\usepackage{fancyhdr}
\pagestyle{fancy}
\fancyhf{}
\fancyhead[L]{\raisebox{0.20cm}{\textbf{Finite Elements - IOC5107}}}
\fancyhead[R]{\raisebox{0.1cm}{\includegraphics[width=0.25\linewidth]{LOGO_UNIVERSIDAD.jpg}}}
\fancyhead[C]{\rule{\textwidth}{0.6pt}}
\fancyfoot[C]{\rule{\textwidth}{0.6pt}}
\fancyfoot[R]{\raisebox{-1.5\baselineskip}{\thepage}}
\renewcommand{\headrulewidth}{0pt}
\renewcommand{\footrulewidth}{0pt}

% Configuración avanzada de geometría
\geometry{
  top=3.5cm, % Aumenta el espacio en la parte superior para subir el encabezado
  bottom=2.5cm,
  headheight=2.5cm % Aumenta la altura del encabezado si es necesario
}

% Configuracion de bibliografia
\usepackage{natbib}
\bibliographystyle{unsrtnat}  % Puedes cambiarlo por `unsrtnat`, `abbrvnat`, etc.

\begin{document}

%----------------------------------------------------------------------------------------
% PORTADA
%----------------------------------------------------------------------------------------
\begin{titlepage}
\newcommand{\HRule}{\rule{\linewidth}{0.5mm}} 
\center 
%----------------------------------------------------------------------------------------
%	ENCABEZADO
%----------------------------------------------------------------------------------------
\includegraphics[width=10cm]{LOGO_UNIVERSIDAD.jpg}\\ 
\vspace{3cm}
%----------------------------------------------------------------------------------------
%	SECCION DEL TITULO
%----------------------------------------------------------------------------------------
\HRule \\[0.4cm]
{ \huge \bfseries Finite Elements - IOC5107}\\[0.4cm] 
{ \huge \bfseries Final Report Homework 4}\\[0.4cm] 
\HRule \\[1.5cm]
 \vspace{5cm}
%----------------------------------------------------------------------------------------
%	SECCION DEL AUTOR
%----------------------------------------------------------------------------------------
\begin{flushright}
  { \textbf{Profesor:}\\
  José Antonio Abell\\
  \textbf{Ayudante:}\\
  Nicolás Mora\\
  \textbf{Alumnos:} \\
  Felipe Vicencio\\
  Lukas Wolff\\
}
\end{flushright}
\vspace{1cm}
%----------------------------------------------------------------------------------------
%	SECCION DE LA FECHA
%----------------------------------------------------------------------------------------
{\large \textbf{\today}}\\[2cm] 
\end{titlepage}

\newpage
\thispagestyle{empty}

\setcounter{page}{1}

%----------------------------------------------------------------------------------------
%ACÁ EMPIEZA EL INFORME
%----------------------------------------------------------------------------------------
\section{Introduction}

This report extends the finite‐element study of Poisson’s equation in two dimensions by building a complete FEM solver that reads an unstructured triangular mesh, assembles and solves the variational problem using both Constant Strain Triangle (CST) and Linear Serendipity Triangle (LST) elements, and rigorously verifies its accuracy.  The solver begins by parsing node coordinates, element connectivities and physical tags from the mesh file, thereby capturing the spatially varying element size $h$ directly as defined by the mesh generator.  Using inner‐product and Sobolev‐space theory, the Poisson problem with Dirichlet boundary conditions is recast into its weak form and discretized via the Galerkin method.  Element‐by‐element stiffness matrices are computed by numerical integration of the gradients of the shape functions, then assembled into the global system, to which boundary values are applied and the resulting linear system is solved.

To verify correctness and quantify convergence, the Method of Manufactured Solutions (MMS) is employed.  An analytic $u_{\mathrm{MMS}}$ is prescribed, its Laplacian yields a source term $f=-\Delta u_{\mathrm{MMS}}$, and Dirichlet boundary data $g=u_{\mathrm{MMS}}|_{\partial\Omega}$ are enforced.  The solver is run on a sequence of uniformly refined and non‐uniformly refined meshes; the discrete solution $u_h$ is compared to $u_{\mathrm{MMS}}$ in the $L^2$ and $H^1$ norms, and $\log$–$\log$ plots of error versus $h$ are used to extract observed convergence rates, which should match the theoretical orders ($O(h)$ and $O(h^2)$ in $H^1$, and one order higher in $L^2$ for CST and LST respectively).

This assignment thus integrates mesh handling, variational theory, element assembly, boundary condition enforcement, solver implementation, and systematic code verification, providing a thorough hands‐on experience in modern finite‐element analysis.  

\section{Theoretical Background}

This section gathers the mathematical foundations and numerical techniques used to formulate, discretize, and verify the solution of the Poisson problem via the Finite Element Method (FEM).

\subsection{Inner Product Spaces and Norms}

Inner product spaces provide the abstract setting for variational formulations and error analysis.  An inner product space is a real vector space $V$ endowed with a bilinear form
\begin{equation}
  (u,v) = \langle u,v\rangle,
\end{equation}
which is symmetric, linear in its first argument, and positive‐definite.  It induces the norm
\begin{equation}
  \|v\| = \sqrt{\langle v,v\rangle},
\end{equation}
and satisfies the Cauchy–Schwarz inequality $|\langle u,v\rangle|\le\|u\|\|v\|$.  Completeness under this norm yields a Hilbert space structure, essential for Lax–Milgram arguments \cite{inner_product_space}.

\subsection{Sobolev Spaces $H^1$ and $H_0^1$}

Sobolev spaces extend inner product concepts to functions with weak derivatives.  The space
\begin{equation}
  H^1(\Omega)
  = \{\,v\in L^2(\Omega)\mid \nabla v\in [L^2(\Omega)]^d\}
\end{equation}
carries the norm
\begin{equation}
  \|v\|_{H^1}^2 = \|v\|_{L^2}^2 + \|\nabla v\|_{L^2}^2.
\end{equation}
Imposing homogeneous Dirichlet data leads to
\begin{equation}
  H_0^1(\Omega)
  = \overline{C_c^\infty(\Omega)}^{\|\cdot\|_{H^1}},
\end{equation}
the natural trial space for PDEs with zero boundary conditions \cite{sobolev_space}.

\subsection{Hilbert Spaces}

A Hilbert space is an inner product space that is complete with respect to the norm induced by its inner product.  Completeness means that every Cauchy sequence $\{v_n\}\subset V$ satisfies
\begin{equation}
  \lim_{m,n\to\infty}\|v_n - v_m\| = 0
  \quad\Longrightarrow\quad
  \exists\,v\in V:\ \lim_{n\to\infty}\|v_n - v\| = 0.
\end{equation}
Key examples include $L^2(\Omega)$ with inner product
\begin{equation}
  (u,v)_{L^2} = \int_\Omega u\,v\,dx,
\end{equation}
and the Sobolev space $H^1(\Omega)$ itself.  The Riesz representation theorem in a Hilbert space $H$ states that every continuous linear functional $F:H\to\mathbb{R}$ can be written uniquely as
\begin{equation}
  F(v) = (u_F, v)_H
  \quad\text{for some }u_F\in H.
\end{equation}
This result underpins the variational theory of PDEs, since the mapping $v\mapsto\ell(v)$ in the weak formulation can be identified with an element of $H_0^1(\Omega)$, guaranteeing existence and uniqueness of solutions. \cite{inner_product_space}

\subsection{Poisson Problem in 2D}

The Poisson equation models steady‐state diffusion or potential fields.  In two dimensions:
\begin{equation}
  -\Delta u = f \quad \text{in}\ \Omega,
\end{equation}
subject to boundary conditions on $\partial\Omega$.  Here, $\Delta u=\partial_{xx}u+\partial_{yy}u$ and $f$ is a source term.  Dirichlet conditions $u=g$ or Neumann conditions $\partial_n u=h$ prescribe values on $\partial\Omega$ \cite{poisson_equation, dirichlet_boundary_condition, neumann_boundary_condition}.

\subsection{Weak (Variational) Formulation}

Rewriting the boundary‐value problem in Sobolev spaces allows FEM discretization.  For $u-g\in H_0^1(\Omega)$, multiply by test $v\in H_0^1(\Omega)$ and integrate by parts:
\begin{equation}
  \int_\Omega \nabla u\cdot\nabla v\,dx
  = \int_\Omega f\,v\,dx
  - \int_{\partial\Omega} h\,v\,ds.
\end{equation}
Defining
\begin{equation}
  a(u,v)=\int_\Omega\nabla u\cdot\nabla v\,dx,\quad
  \ell(v)=\int_\Omega f\,v\,dx-\int_{\partial\Omega}h\,v\,ds,
\end{equation}
the problem becomes: find $u\in H^1(\Omega)$ such that
\begin{equation}
  a(u,v)=\ell(v)\quad\forall\,v\in H_0^1(\Omega).
\end{equation}
Continuity and coercivity of $a(\cdot,\cdot)$ guarantee a unique solution via Lax–Milgram \cite{weak_formulation}.

\subsection{Galerkin Method}

The Galerkin method projects the infinite‐dimensional weak problem onto a finite subspace $V_h\subset H_0^1(\Omega)$.  One seeks $u_h\in V_h$ such that
\begin{equation}
  a(u_h,v_h)=\ell(v_h)
  \quad\forall\,v_h\in V_h.
\end{equation}
This ensures \emph{Galerkin orthogonality}
\begin{equation}
  a(u-u_h,v_h)=0
  \quad\forall\,v_h\in V_h,
\end{equation}
which is fundamental to derive error estimates \cite{Galerkin_method}.

\subsection{Finite Element Spaces: CST and LST}

Finite element spaces consist of piecewise‐defined basis functions over a mesh.

\subsubsection{CST Elements (Constant Strain Triangle)}

CST uses three linear shape functions per triangle $T$, each associated with a vertex and satisfying $\varphi_i(x_j)=\delta_{ij}$.  The stiffness matrix entry is
\begin{equation}
  K^e_{ij}=\int_T\nabla\varphi_i\cdot\nabla\varphi_j\,dx,
\end{equation}
with constant gradients on $T$, yielding a simple, first‐order accurate scheme \cite{finite_element_method}.

\subsubsection{LST Elements (Linear Serendipity Triangle)}

LST augments CST by adding three mid‐edge nodes, producing six quadratic shape functions per triangle.  The same formula
\begin{equation}
  K^e_{ij}=\int_T\nabla\varphi_i\cdot\nabla\varphi_j\,dx
\end{equation}
applies, but gradients vary within $T$, giving second‐order convergence at increased computational cost \cite{finite_element_method}.

\subsection{Manufactured Solution Method}

The MMS provides a systematic code verification test.  One selects an analytic $u_{\rm MMS}$, then computes
\begin{equation}
  f=-\Delta u_{\rm MMS},
  \quad
  g=u_{\rm MMS}\big|_{\partial\Omega}.
\end{equation}
Solving the FEM system with these data and comparing $u_h$ to $u_{\rm MMS}$ in various norms reveals implementation errors \cite{Method_of_manufactured_solutions}.

\subsection{Convergence Study}

Error analysis predicts for CST
\begin{equation}
  \|u-u_h\|_{H^1}=O(h),
  \quad
  \|u-u_h\|_{L^2}=O(h^2),
\end{equation}
and for LST
\begin{equation}
  \|u-u_h\|_{H^1}=O(h^2),
  \quad
  \|u-u_h\|_{L^2}=O(h^3).
\end{equation}
Numerical experiments on successive mesh refinements, plotted as $\log(\|e\|)$ vs.\ $\log(h)$, confirm these rates \cite{Convergence_numerical_analysis}.


\section{Results}

In this section we present the numerical results obtained for the Poisson problem model using both linear (CST) and quadratic (LST) triangular elements. We generate a sequence of uniform and geometrically graded meshes (controlled by the progression parameter $r$) and assemble the corresponding finite‐element systems by exact integration of the element stiffness matrices. After enforcing homogeneous Dirichlet conditions, we solve each sparse linear system to obtain the discrete solution \(u_h\). We then compute the energy‐norm error \(\|u - u_h\|_{H^1(\Omega)}\) against the known analytic solution and plot its decay versus the mesh size \(h\) on a log–log scale. By fitting a straight line to these data, we extract the observed convergence rates and compare them to the theoretical predictions of \(\mathcal{O}(h^1)\) for CST and \(\mathcal{O}(h^2)\) for LST. Tables and figures below summarize these findings for both uniform and graded discretizations.  

\subsection{CST annalysis}

\begin{figure}[H]
\centering
\includegraphics[width=0.4\textwidth]{GRAFICOS/CST/CST_mesh_plot.png}
\caption{CST Mesh}
\label{fig:cst_results}
\end{figure}

\begin{figure}[H]
  \centering
  \begin{subfigure}[b]{0.48\textwidth}
    \centering
    \includegraphics[width=\textwidth]{GRAFICOS/CST/CST_u_fem_sol_surface_plot.png}
    \caption{Discrete solution \(u_h\) for CST}
    \label{fig:cst_u_fem_sol}
  \end{subfigure}
  \hfill
  \begin{subfigure}[b]{0.48\textwidth}
    \centering
    \includegraphics[width=\textwidth]{GRAFICOS/CST/CST_u_sol_surface_plot.png}
    \caption{Analytic solution \(u\) for CST}
    \label{fig:cst_error_plot}
  \end{subfigure}
  \caption{Comparison of the finite‐element discrete solution \(u_h\) and the analytic solution \(u\) for CST elements.}
  \label{fig:cst_comparison}
\end{figure}

\begin{figure}[H]
\centering
\includegraphics[width=0.6\textwidth]{GRAFICOS/CST/CST_relative_error_surface_plot.png}
\caption{Relative error \(\|u - u_h\|_{H^1(\Omega)}\) for CST elements}
\label{fig:cst_error_vs_h}
\end{figure}


\subsection{LST annalysis}

\begin{figure}[H]
\centering
\includegraphics[width=0.4\textwidth]{GRAFICOS/LST/LST_mesh_plot.png}
\caption{LST Mesh}
\label{fig:lst_results}
\end{figure}

\begin{figure}[H]
\centering
\begin{subfigure}[b]{0.48\textwidth}
    \centering
    \includegraphics[width=\textwidth]{GRAFICOS/LST/LST_u_fem_sol_surface_plot.png}
    \caption{Discrete solution \(u_h\) for LST}
    \label{fig:lst_u_fem_sol}
  \end{subfigure}
  \hfill
  \begin{subfigure}[b]{0.48\textwidth}
    \centering
    \includegraphics[width=\textwidth]{GRAFICOS/LST/LST_u_sol_surface_plot.png}
    \caption{Analytic solution \(u\) for LST}
    \label{fig:lst_error_plot}
  \end{subfigure}
  \caption{Comparison of the finite‐element discrete solution \(u_h\) and the analytic solution \(u\) for LST elements.}
  \label{fig:lst_comparison}
\end{figure}

\begin{figure}[H]
\centering
\includegraphics[width=0.6\textwidth]{GRAFICOS/LST/LST_relative_error_surface_plot.png}
\caption{Relative error \(\|u - u_h\|_{H^1(\Omega)}\) for LST elements}
\label{fig:lst_error_vs_h}
\end{figure}

\section{Conclusions}

In this report we have developed and rigorously verified a finite element solver for the two dimensional Poisson equation on unstructured triangular meshes, using both Constant Strain Triangle (CST) and Linear Serendipity Triangle (LST) elements. By applying the Method of Manufactured Solutions, we derived the exact source term and Dirichlet data, assembled element stiffness matrices via Gaussian quadrature, and measured the discrete error $\|u - u_h\|_{H^1(\Omega)}$ over a sequence of uniform and geometrically graded meshes ($r=1.00,1.05,1.10$).

On uniform meshes ($r=1.00$), CST exhibited an empirical error decay $E(N)\sim N^{-2.16}$, confirming second order pointwise convergence of linear elements, while LST achieved its theoretical second order rate in the convergence. Geometric grading produced a transient exponential convergence regime, driven by the exponential shrinkage of the smallest element sizes $h_{\min}\sim r^{-N}$, with observed slopes up to $p\approx3.1$ before machine precision was reached. Beyond this, the global maximum error was controlled by the largest elements ($h_{\max}\to(r-1)/r$), causing the convergence curves to rise and flattens.

Overall, CST elements deliver $\|e\|_{H^1}=O(h)$, $\|e\|_{L^2}=O(h^2)$, and LST elements deliver $\|e\|_{H^1}=O(h^2)$, $\|e\|_{L^2}=O(h^3)$, with mesh grading effectively reducing error constants but not altering asymptotic orders. These findings validate both the mathematical derivations and the software implementation, and illustrate the trade off between targeted error reduction via grading and the emergence of a global error floor determined by the coarsest elements.



\bibliography{ref}

\end{document}
