\section{Theoretical Background}

This section gathers the mathematical foundations and numerical techniques used to formulate, discretize, and verify the solution of the Poisson problem via the Finite Element Method (FEM).

\subsection{Inner Product Spaces and Norms}

Inner product spaces provide the abstract setting for variational formulations and error analysis.  An inner product space is a real vector space $V$ endowed with a bilinear form
\begin{equation}
  (u,v) = \langle u,v\rangle,
\end{equation}
which is symmetric, linear in its first argument, and positive definite.  It induces the norm
\begin{equation}
  \|v\| = \sqrt{\langle v,v\rangle},
\end{equation}
and satisfies the Cauchy–Schwarz inequality $|\langle u,v\rangle|\le\|u\|\|v\|$.  Completeness under this norm yields a Hilbert space structure, essential for Lax–Milgram arguments. \citep{Rudin1973,Brezis2011}

\subsection{Sobolev Spaces $H^1$ and $H_0^1$}

Sobolev spaces extend inner product concepts to functions with weak derivatives.  The space
\begin{equation}
  H^1(\Omega)
  = \{\,v\in L^2(\Omega)\mid \nabla v\in [L^2(\Omega)]^d\}
\end{equation}
carries the norm
\begin{equation}
  \|v\|_{H^1}^2 = \|v\|_{L^2}^2 + \|\nabla v\|_{L^2}^2.
\end{equation}
Imposing homogeneous Dirichlet data leads to
\begin{equation}
  H_0^1(\Omega)
  = \overline{C_c^\infty(\Omega)}^{\|\cdot\|_{H^1}},
\end{equation}
the natural trial space for PDEs with zero boundary conditions. \citep{Adams1975,Brezis2011}

\subsection{Hilbert Spaces}

A Hilbert space is an inner product space that is complete with respect to the norm induced by its inner product.  Completeness means that every Cauchy sequence $\{v_n\}\subset V$ satisfies
\begin{equation}
  \lim_{m,n\to\infty}\|v_n - v_m\| = 0
  \quad\Longrightarrow\quad
  \exists\,v\in V:\ \lim_{n\to\infty}\|v_n - v\| = 0.
\end{equation}
Key examples include $L^2(\Omega)$ with inner product
\begin{equation}
  (u,v)_{L^2} = \int_\Omega u\,v\,dx,
\end{equation}
and the Sobolev space $H^1(\Omega)$ itself.  The Riesz representation theorem in a Hilbert space $H$ states that every continuous linear functional $F:H\to\mathbb{R}$ can be written uniquely as
\begin{equation}
  F(v) = (u_F, v)_H
  \quad\text{for some }u_F\in H.
\end{equation}
This result underpins the variational theory of PDEs, since the mapping $v\mapsto\ell(v)$ in the weak formulation can be identified with an element of $H_0^1(\Omega)$, guaranteeing existence and uniqueness of solutions. \citep{Brezis2011}

\subsection{Poisson Problem in 2D}

The Poisson equation models steady state diffusion or potential fields.  In two dimensions:
\begin{equation}
  -\Delta u = f \quad \text{in}\ \Omega,
\end{equation}
subject to boundary conditions on $\partial\Omega$.  Here, $\Delta u=\partial_{xx}u+\partial_{yy}u$ and $f$ is a source term.  Dirichlet conditions $u=g$ or Neumann conditions $\partial_n u=h$ prescribe values on $\partial\Omega$. \citep{Evans2010}

\subsection{Weak (Variational) Formulation}

Rewriting the boundary value problem in Sobolev spaces allows FEM discretization.  For $u-g\in H_0^1(\Omega)$, multiply by test $v\in H_0^1(\Omega)$ and integrate by parts:
\begin{equation}
  \int_\Omega \nabla u\cdot\nabla v\,dx
  = \int_\Omega f\,v\,dx
  - \int_{\partial\Omega} h\,v\,ds.
\end{equation}
Defining
\begin{equation}
  a(u,v)=\int_\Omega\nabla u\cdot\nabla v\,dx,\quad
  \ell(v)=\int_\Omega f\,v\,dx-\int_{\partial\Omega}h\,v\,ds,
\end{equation}
the problem becomes: find $u\in H^1(\Omega)$ such that
\begin{equation}
  a(u,v)=\ell(v)\quad\forall\,v\in H_0^1(\Omega).
\end{equation}
Continuity and coercivity of $a(\cdot,\cdot)$ guarantee a unique solution via Lax–Milgram. \citep{Ciarlet1978}

\subsection{Galerkin Method}

The Galerkin method projects the infinite‐dimensional weak problem onto a finite subspace $V_h\subset H_0^1(\Omega)$.  One seeks $u_h\in V_h$ such that
\begin{equation}
  a(u_h,v_h)=\ell(v_h)
  \quad\forall\,v_h\in V_h.
\end{equation}
This ensures \emph{Galerkin orthogonality}
\begin{equation}
  a(u-u_h,v_h)=0
  \quad\forall\,v_h\in V_h,
\end{equation}
which is fundamental to derive error estimates. \citep{ErnGuermond2004}

\subsection{Gauss Quadrature}

Numerical integration of stiffness and load integrals in FEM is typically carried out by Gauss (or Gauss–Legendre) quadrature.  In one dimension, the $n$‐point Gauss rule on the reference interval $[-1,1]$ is
\begin{equation}
  \int_{-1}^{1} f(x)\,dx
  \;\approx\;
  \sum_{i=1}^{n} w_i\,f(x_i),
\end{equation}
where the nodes $x_i$ are the roots of the Legendre polynomial $P_n(x)$ and the weights $w_i$ are chosen so that the rule is exact for all polynomials of degree up to $2n-1$.  Concretely,
\begin{equation}
  w_i = \frac{2}{\bigl(1 - x_i^2\bigr)\bigl[P_n'(x_i)\bigr]^2}\,,
  \quad i = 1,\dots,n.
\end{equation}

To integrate over a general interval $[a,b]$, one applies the affine mapping
\[
  x = \frac{b-a}{2}\,\xi + \frac{a+b}{2}, 
  \quad \xi\in[-1,1],
\]
yielding
\begin{equation}
  \int_a^b f(x)\,dx
  = \frac{b-a}{2}
    \sum_{i=1}^{n} w_i\,f\!\Bigl(\tfrac{b-a}{2}x_i + \tfrac{a+b}{2}\Bigr).
\end{equation}

For two dimensional triangles $T$, one can either use a tensor product of 1D rules on a reference square and map via an isoparametric transformation, or employ specialized triangular rules with nodes $\{\xi_q\}$ and weights $\{w_q\}$ satisfying
\begin{equation}
  \int_{T} f(x,y)\,dx\,dy
  \;\approx\;
  \sum_{q=1}^{N_q} w_q\,f\bigl(x(\xi_q),\,y(\xi_q)\bigr),
\end{equation}

exact for polynomials up to a given degree.  In P1 (CST) elements, a one or three point rule is sufficient since $\nabla\varphi_i\cdot\nabla\varphi_j$ is constant; for P2 (LST), one uses at least a seven point rule to exactly integrate up to quartic terms in the reference triangle.  

Proper choice of quadrature order ensures both accuracy of the stiffness matrix assembly and preservation of the FEM convergence rates. \citep{Dunavant1985}

\subsection{Finite Element Spaces: CST and LST}

Finite element spaces consist of piecewise defined basis functions over a mesh.

\subsubsection{CST Elements (Constant Strain Triangle)}

CST uses three linear shape functions per triangle $T$, each associated with a vertex and satisfying $\varphi_i(x_j)=\delta_{ij}$.  The stiffness matrix entry is
\begin{equation}
  K^e_{ij}=\int_T\nabla\varphi_i\cdot\nabla\varphi_j\,dx,
\end{equation}
with constant gradients on $T$, yielding a simple, first order accurate scheme. \citep{Zienkiewicz2005}

\subsubsection{LST Elements (Linear Serendipity Triangle)}

LST augments CST by adding three mid edge nodes, producing six quadratic shape functions per triangle.  The same formula
\begin{equation}
  K^e_{ij}=\int_T\nabla\varphi_i\cdot\nabla\varphi_j\,dx
\end{equation}
applies, but gradients vary within $T$, giving second order convergence at increased computational cost. \citep{Zienkiewicz2005}

\subsection{Manufactured Solution Method}

The MMS provides a systematic code verification test.  One selects an analytic $u_{\rm MMS}$, then computes
\begin{equation}
  f=-\Delta u_{\rm MMS},
  \quad
  g=u_{\rm MMS}\big|_{\partial\Omega}.
\end{equation}
Solving the FEM system with these data and comparing $u_h$ to $u_{\rm MMS}$ in various norms reveals implementation errors. \citep{Roache1998}

\subsection{Convergence Study}

Error analysis predicts for CST
\begin{equation}
  \|u-u_h\|_{H^1}=O(h),
  \quad
  \|u-u_h\|_{L^2}=O(h^2),
\end{equation}
and for LST
\begin{equation}
  \|u-u_h\|_{H^1}=O(h^2),
  \quad
  \|u-u_h\|_{L^2}=O(h^3).
\end{equation}
Numerical experiments on successive mesh refinements, plotted as $\log(\|e\|)$ vs.\ $\log(h)$, confirm these rates. \citep{Johnson2009}
