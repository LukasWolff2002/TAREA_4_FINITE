\section{Theoretical Background}

In this section, the theoretical background related to solving the Poisson problem using the Finite Element Method (FEM), as well as the verification method known as the \textit{Manufactured Solution Method} (MMS), is presented. The goal is to provide a detailed understanding of the necessary fundamentals to approach HW4.

\subsection{Poisson Problem in 2D}

The Poisson problem is a second-order partial differential equation (PDE) widely used in physics and engineering. Its general form is:

\begin{equation}
    -\Delta u = f \quad \text{in} \quad \Omega,
\end{equation}
with boundary conditions on the domain \(\partial \Omega\), where:
- \(\Delta\) is the Laplace operator,
- \(u\) is the unknown function (e.g., electric potential, temperature, etc.),
- \(f\) is the source function (e.g., charge density in electrostatics) \cite{poisson_equation}.

In 2D Cartesian coordinates, the Laplace operator is expressed as:
\begin{equation}
    \Delta u = \frac{\partial^2 u}{\partial x^2} + \frac{\partial^2 u}{\partial y^2}.
\end{equation}
This operator is commonly used to model physical phenomena like heat conduction or electrostatic potential \cite{laplace_operator}.

Boundary conditions can be either Dirichlet or Neumann:
- Dirichlet: \(u(x,y) = g(x,y)\) on \(\partial \Omega\),
- Neumann: \(\frac{\partial u}{\partial n} = h(x,y)\) on \(\partial \Omega\), where \(n\) is the normal direction to the boundary \cite{boundary_conditions}.

In this task, we use the Dirichlet boundary condition \(u|_{\partial \Omega} = g\).

\subsection{Weak Formulation of the Problem}

To apply the Finite Element Method, the problem needs to be transformed into its \textit{weak formulation}. This is done by multiplying the differential equation by a test function \(v(x,y)\) and then integrating by parts. The weak formulation of Poisson is obtained by following these steps:

\begin{equation}
    \int_\Omega -\Delta u \, v \, dx \, dy = \int_\Omega f \, v \, dx \, dy.
\end{equation}

Next, we perform integration by parts on the first term:

\begin{equation}
    \int_\Omega -\Delta u \, v \, dx \, dy = \int_{\partial \Omega} \frac{\partial u}{\partial n} \, v \, ds - \int_\Omega \nabla u \cdot \nabla v \, dx \, dy.
\end{equation}

Where \(\frac{\partial u}{\partial n}\) is the normal derivative of \(u\) on the boundary. The weak formulation of the Poisson problem is:

\begin{equation}
    \int_\Omega \nabla u \cdot \nabla v \, dx \, dy = \int_\Omega f \, v \, dx \, dy - \int_{\partial \Omega} h \, v \, ds,
\end{equation}
for all \(v \in H_0^1(\Omega)\), where \(H_0^1(\Omega)\) is the space of functions with weak derivatives in \(L^2(\Omega)\) and that vanish on the boundary \(\partial \Omega\) \cite{weak_formulation}.

\subsection{Galerkin Method}

The Galerkin Method is a technique used to solve differential equations using approximations based on finite function spaces. The Galerkin principle is applied to the weak formulation of the problem, where we seek an approximation \(u_h \in V_h \subset H_0^1(\Omega)\) such that:

\begin{equation}
    a(u_h, v_h) = \ell(v_h) \quad \forall v_h \in V_h.
\end{equation}

Where:
\begin{equation}
    a(u_h, v_h) = \int_\Omega \nabla u_h \cdot \nabla v_h \, dx,
\end{equation}
and
\begin{equation}
    \ell(v_h) = \int_\Omega f \, v_h \, dx - \int_{\partial \Omega} h \, v_h \, ds.
\end{equation}

Here, \(V_h\) is a finite subspace of \(H_0^1(\Omega)\), typically generated by linear (CST) or quadratic (LST) shape functions, depending on the type of finite elements used \cite{galerkin_method}.

\subsection{Finite Elements: CST and LST}

\subsubsection{CST Elements (Constant Strain Triangle)}

The CST elements (Constant Strain Triangle) are first-order finite elements that use linear shape functions. In 2D, each triangle has three nodes, one at each vertex. The shape functions for these elements are linear and associated with the vertices of the triangle. The stiffness matrix for these elements is computed as:

\begin{equation}
    K^e_{ij} = \int_T \nabla \varphi_i \cdot \nabla \varphi_j \, dx,
\end{equation}
where \(T\) is the triangle and \(\varphi_i\) are the linear shape functions associated with the vertices of the triangle. The matrix of stiffness for CST elements is relatively simple to compute since the gradients of the shape functions are constant within each triangle.

These elements are suitable for problems where the solution is smooth but may not be as accurate near singularities. The CST elements are widely used for problems that do not have sharp gradients or significant non-linearity.

\subsubsection{LST Elements (Linear Serendipity Triangle)}

The LST elements (Linear Serendipity Triangle) are second-order finite elements that use quadratic shape functions. In 2D, each triangle has six nodes: three at the vertices and three at the midpoints of the edges. These shape functions are quadratic and provide more flexibility in approximating the solution. The stiffness matrix for these elements is more complex to compute than for CST elements because the gradients of the shape functions are not constant.

\[
K^e_{ij} = \int_T \nabla \varphi_i \cdot \nabla \varphi_j \, dx,
\]
where \(T\) is the triangle and \(\varphi_i\) are the quadratic shape functions associated with the six nodes. These elements provide a higher degree of accuracy than CST elements, especially for problems with smooth solutions or problems requiring a finer approximation.

LST elements are more computationally expensive but yield more accurate results in simulations requiring high precision.

\subsection{Manufactured Solution Method}

The Manufactured Solution Method (MMS) is a technique used to verify the accuracy of a numerical code. In this method, an exact solution \(u_{\text{MMS}}(x)\) for the Poisson problem is chosen, and the corresponding source term \(f(x)\) and boundary conditions \(g(x)\) or \(h(x)\) are computed to make \(u_{\text{MMS}}(x)\) the exact solution of the differential equation.

The steps to apply MMS are as follows:

\begin{enumerate}
    \item \textbf{Choose an exact solution \(u_{\text{MMS}}(x)\)}: Select a smooth and analytical function as the solution.
    \item \textbf{Calculate the source term \(f(x)\)}: Differentiate the Poisson equation with \(u_{\text{MMS}}(x)\) to obtain \(f(x) = -\Delta u_{\text{MMS}}(x)\).
    \item \textbf{Determine the boundary conditions \(g(x)\) or \(h(x)\)}: If Dirichlet, set \(g(x) = u_{\text{MMS}}(x)\) on the boundary; if Neumann, calculate \(h(x) = \frac{\partial u_{\text{MMS}}}{\partial n}\) on the boundary.
    \item \textbf{Solve the problem numerically}: Use the numerical code to solve the problem with the source term \(f(x)\) and boundary conditions \(g(x)\) or \(h(x)\).
    \item \textbf{Compare the numerical solution with the exact solution \(u_{\text{MMS}}(x)\)}: Compute the error \(e(x) = u_{\text{MMS}}(x) - u_h(x)\) at each node or point on the mesh, and measure the error in norms such as \(L^2\) or \(H^1\).
\end{enumerate}

This method allows us to verify the accuracy of the numerical code and calculate the convergence rate by observing how the error decreases as the mesh is refined \cite{manufactured_solution_method}.

\subsection{Convergence Study}

The convergence study is carried out to verify that the numerical code follows the theoretical convergence expectations. Generally, for first-order finite elements CST, the error in the seminorm \(H^1\) should decay at a rate of \(O(h)\), while the error in the \(L^2\) norm should decay at a rate of \(O(h^2)\). For second-order finite elements LST, we expect the error in \(H^1\) to decay at \(O(h^2)\) and the error in \(L^2\) to decay at \(O(h^3)\).

To study the convergence, several problems are solved with different mesh sizes \(h\), and the \(L^2\) and \(H^1\) norms of the error are computed. The results are plotted on a log-log graph, where the slope of the fitted line indicates the convergence rate \cite{convergence_study}.
