\section{Introduction}

This report extends the finite‐element study of Poisson’s equation in two dimensions by building a complete FEM solver that reads an unstructured triangular mesh, assembles and solves the variational problem using both Constant Strain Triangle (CST) and Linear Serendipity Triangle (LST) elements, and rigorously verifies its accuracy.  The solver begins by parsing node coordinates, element connectivities and physical tags from the mesh file, thereby capturing the spatially varying element size $h$ directly as defined by the mesh generator.  Using inner‐product and Sobolev‐space theory, the Poisson problem with Dirichlet boundary conditions is recast into its weak form and discretized via the Galerkin method.  Element‐by‐element stiffness matrices are computed by numerical integration of the gradients of the shape functions, then assembled into the global system, to which boundary values are applied and the resulting linear system is solved.

To verify correctness and quantify convergence, the Method of Manufactured Solutions (MMS) is employed.  An analytic $u_{\mathrm{MMS}}$ is prescribed, its Laplacian yields a source term $f=-\Delta u_{\mathrm{MMS}}$, and Dirichlet boundary data $g=u_{\mathrm{MMS}}|_{\partial\Omega}$ are enforced.  The solver is run on a sequence of uniformly refined and non‐uniformly refined meshes; the discrete solution $u_h$ is compared to $u_{\mathrm{MMS}}$ in the $L^2$ and $H^1$ norms, and $\log$–$\log$ plots of error versus $h$ are used to extract observed convergence rates, which should match the theoretical orders ($O(h)$ and $O(h^2)$ in $H^1$, and one order higher in $L^2$ for CST and LST respectively).

This assignment thus integrates mesh handling, variational theory, element assembly, boundary condition enforcement, solver implementation, and systematic code verification, providing a thorough hands‐on experience in modern finite‐element analysis.  
